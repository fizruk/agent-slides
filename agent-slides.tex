\documentclass{beamer}
\usepackage[utf8]{inputenc}
\usepackage[english,russian]{babel}
\usepackage{hyperref}
\usepackage{xcolor}
\usepackage{graphicx}

\usetheme{Warsaw}
\setbeamercovered{transparent}% Allow for shaded (transparent) covered items

\begin{document}

\title[]{Многоагентные системы}
\author{Н.\,Д.~Кудасов}

\institute{
    % {\em Научный руководитель}:\\
    \vspace{2.0cm}
}
\date{Москва, 2013}

\begin{frame}
\addtocounter{framenumber}{-1}
\maketitle
\end{frame}

\section{Агенты}

\begin{frame}
\frametitle{Что такое агент?}
  Агенты часто используются в области ИИ при разработке систем.
  Тем не менее, до сих пор не существует устоявшегося понятия.

  \begin{exampleblock}{}
    {\large ``Агент ~--- это инкапсулированная вычислительная система,
    помещенная в некоторую среду и способная автономно выполнять действия
    в этой среде для достижения поставленных целей.''}
    \vskip5mm
    \hspace*\fill{\small--- Wooldridge and Jennings}
  \end{exampleblock}

  % TODO: add distinct definition
\end{frame}

\begin{frame}
\frametitle{Общее понятие}
  В самом общем случае агент обычно представляет нечто, обладающее
  следующими свойствами:

  \begin{itemize}
    \item<1-> {\it реакционность}: способность агента воспринимать окружающее и влиять на него;
    \item<2-> {\it целеустремленность}: агент должен действовать в заложенными в него целями;
    \item<3-> {\it социальная активность}: агент должен взаимодействовать с другими агентами и/или людьми;
    \item<4-> {\it автономность}: агент действует без непосредственного вмешательства человека и обладает
      определенным контролем на своими действиями и внутренним состоянием.
  \end{itemize}
\end{frame}

\begin{frame}
\frametitle{Ментальность}
  Распространено использование ментальных характеристик:

  \begin{itemize}
    \item знания,
    \item убеждения,
    \item намерения,
    \item обязательства и т.п.
  \end{itemize}

  Иногда агенты наделяются {\it эмоциями}.
\end{frame}

\begin{frame}
\frametitle{Прочие свойства}
  Часто также обсуждаются следующие свойства агентов:

  \begin{itemize}
    \item {\it мобильность}: способность агентов перемещаться (физически или в сети);
    \item {\it правдивость}: предположение, что агент не может намеренно фальсифицировать передаваемую информацию;
    \item {\it доброжелательность}: предположение, что цели агентов не конфликтуют и, следовательно, каждый агент
      стремиться выполнить то, о чём его просят;
    \item {\it рациональность}: предположение, что агент действует в соответствии со своими целями и не пытается
      противостоять себе (по крайней мере, насколько это позволяют его убеждения).
  \end{itemize}
\end{frame}

\begin{frame}
\frametitle{Концепции агентов}
  Существует ряд формальных теорий, описывающих агентов:

  \begin{itemize}
    \item {\it логические системы}:
      цели и свойства агента описываются при помощи высказываний
      в различных логических системах;
    \item {\it системы намерений}:
      внутреннее состояние агента представляется системой
      мировоззрений (знания, убеждения, желания, намерения и т.п.);
    \item {\it модели коммуникаций}:
      взаимодействие между агентами происходит посредством специальных
      действий.
  \end{itemize}
\end{frame}

\begin{frame}{}
\addtocounter{framenumber}{-1}
\begin{center}
\LARGE{Спасибо за внимание!}
\end{center}
\end{frame}

\end{document}

