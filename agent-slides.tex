\documentclass{beamer}
\usepackage[utf8]{inputenc}
\usepackage[english,russian]{babel}
\usepackage{hyperref}
\usepackage{xcolor}
\usepackage{graphicx}

\usetheme{Warsaw}

\begin{document}

\title[]{Многоагентные системы}
\author{Н.\,Д.~Кудасов}

\institute{
    % {\em Научный руководитель}:\\
    \vspace{2.0cm}
}
\date{Москва, 2013}

\begin{frame}
\addtocounter{framenumber}{-1}
\maketitle
\end{frame}

\section{Агенты}

\begin{frame}
\frametitle{Что такое агент?}
  Агенты часто используются в области ИИ при разработке систем.
  Тем не менее, до сих пор не существует устоявшегося понятия.

  \begin{exampleblock}{}
    {\large ``Агент ~--- это инкапсулированная вычислительная система,
    помещенная в некоторую среду и способная автономно выполнять действия
    в этой среде для достижения поставленных целей.''}
    \vskip5mm
    \hspace*\fill{\small--- Wooldridge and Jennings}
  \end{exampleblock}

  % TODO: add distinct definition
\end{frame}

\begin{frame}
\frametitle{Общее понятие}
  В самом общем случае агент обычно представляет нечто, обладающее
  следующими свойствами:

  \begin{itemize}
    \item<1-> {\bf реакционность}: способность агента воспринимать окружающее и влиять на него;
    \item<2-> {\bf целеустремленность}: агент должен действовать в заложенными в него целями;
    \item<3-> {\bf социальная активность}: агент должен взаимодействовать с другими агентами и/или людьми;
    \item<4-> {\bf автономность}: агент действует без непосредственного вмешательства человека и обладает
      определенным контролем на своими действиями и внутренним состоянием.
  \end{itemize}
\end{frame}

\begin{frame}
\frametitle{Более строгие понятия}
  На практике часто вводится более строгое понятие, наделяющее агента
  свойствами, которые часто применяются к людям.
  
  Например, распространено использование терминов ``знание'', ``убеждения'',
  ``намерения'' и ``обязательства'' в применение к агенту. Иногда агент даже
  наделяется {\it эмоциями}.
\end{frame}

\begin{frame}
\frametitle{Более строгие понятия}
  На практике часто вводится более строгое понятие, наделяющее агента
  свойствами, которые часто применяются к людям.
  
  Например, распространено использование терминов ``знание'', ``убеждения'',
  ``намерения'' и ``обязательства'' в применение к агенту. Иногда агент даже
  наделяется {\it эмоциями}.
\end{frame}

\begin{frame}{}
\addtocounter{framenumber}{-1}
\begin{center}
\LARGE{Спасибо за внимание!}
\end{center}
\end{frame}

\end{document}

